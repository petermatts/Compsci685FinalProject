\section{Our approach}

\subsection{Proposed Expirement}

\citet{soviany2022curriculumlearningsurvey} found that for NLP tasks curriculum learning is often based on relatively simple heuristics such as text length. For more effective curriculum learning it is proposed that the curriculum should be designed to resemble that of a human learner's experience more closely. This is what we aim to provide for our LLM. We will finetune a LLM by first learning grade school math and arithmetic. We will then build upon this by fine-tuning over more advanced mathematics at the high school level, comprising of algebra, geometry, number theory, and probability problems.

\subsection{Schedule}

Below is our proposed timetable with corresponding descriptions for each of the stages of our project and experiment development.

\begin{center}
\begin{tabular}{cc}
     Task & Time Est. \\\hline
     Data \& Models & 2 Weeks \\
     Framework Setup & 2 Weeks \\
     Initial Experiments & 1 Week \\
     Analysis/Revision & 1 Week \\
     Further Experiments & 2 Weeks \\
     Final Report & 1 Week \\
\end{tabular}
\end{center}

\begin{enumerate}
    \item Data \& Models

    In this stage, we will gather our datasets and select the pretrained LLMs that we intend to use to perform our fine tuning. We will also perform any necessary data pre-processing/formatting that we will need for the data to be compatible with our selected models.
    
    \item Framework Setup

    Our framework setup is the stage in which we will develop our training pipeline for our selected LLMs. This will involve setting up our training functions and parameters as well as our testing functions that will later be used for displaying and evaluating results.
    
    \item Initial Experiments

    At this stage of the project we will begin running our experiments and logging initial results for analysis.
    
    \item Analysis/Revision

    Based on the initial results from the previous stage we will implement any method adjustments and/or any bug fixes that may be necessary. 
    
    \item Further Experiments

    This stage is where we will rerun our experiment with modifications from the previous revisions and obtain our final results.
    
    \item Final Report

    In this stage we will transition from primarily running experiments to converting our results into a presentable format with as much in-depth, detailed explanation as possible. All proposal structure (such as this) will be cleaned up/removed from this report at this time.
\end{enumerate}
